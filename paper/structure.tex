\section{Draft Skeleton}
\begin{itemize} 
\item Introduction and overview. This should frame the problem,
give an overview of everything, and specify exactly what we're specifically
going to focus on, or conversely ignore
\begin{itemize}
\item Shift to multicore and manycore
\begin{itemize}
\item Out of necessity due to the triple hurdle of lack of ILP, power consumption,
and usual blurb. Key point, out of necessity!
\item What is a multi-core/many-core machine, in brief (keeping it brief)?
\begin{itemize}
\item Focus on the 4 architectures that we have
\item Distinction between many core / multi core
\item Distinction between homogeneous cores / heterogeneous cores
\item Coherent memory or not, etc. 
\end{itemize}
\item What are the main consequences of the shift to many-core/multi-core?
\begin{itemize}
\item Heterogeneity across different machines: more diversity in how 
computers are constructed. harder to adapt
\item Heterogeneity within a machine: \\
- When it comes to communication (IPC, interconnects, NUMA) \\ 
- When it comes to the processing power, functionalities of 
different cores \\
\end{itemize}
\item Shifting from a predominantly single threaded model where we can exploit
ILP to a predominantly parallel model where we must predominantly exploit TLP. 
\end{itemize}
\item There's three avenues that need to be considered potentially:
\begin{itemize}
\item Hardware level, how do you design new processors
\item OS level, how do you design your OS for a potentially highly heterogeneous
many-core processor
\item Application, how do you design an application to deal with this, and the high
level of parallelism that one can exist. 
\end{itemize}
\item At each of this three levels, design decisions make an explicit tradeoff
between
\begin{itemize}
\item Power
\item Performance
\item Programmibility
\end{itemize}
\end{itemize}
\item In this paper, we present a taxonomy of the hardware and software
design decisions etc 
\end{itemize}
