
% IEEE standards
% http://conferences.cis.unisa.edu.au/2006/tabletop2006/IEEE/Format/instruct.htm
%All printed material, including text, illustrations, and charts, must be kept within a print area of 6-7/8 inches (17.5 cm) wide by 8-7/8 inches (22.54 cm) high. Do not write or print anything outside the print area. All text must be in a two-column format. Columns are to be 3-1/4 inches (8.25 cm) wide, with a 5/16 inch (0.8 cm) space between them. Text must be fully justified.

\geometry{letterpaper,width=17.5cm,height=23cm} %22.54cm
\setlength{\columnsep}{0.8cm}

\setlength{\parindent}{1pc}

\renewcommand{\baselinestretch}{0.96}

\renewcommand{\topfraction}{.98}
\renewcommand{\topfraction}{.98}
\renewcommand{\textfraction}{.02}
\renewcommand{\bottomfraction}{.7}
\renewcommand{\floatpagefraction}{.66}
\renewcommand{\dbltopfraction}{.98}
\renewcommand{\dblfloatpagefraction}{.66}

\setlength{\textfloatsep}{8pt}
\setlength{\abovecaptionskip}{2pt}

\allowdisplaybreaks

% don't waste so much space on \paragraph headings
\titlespacing{\paragraph}{0in}{0.02in}{0.07in}

\titlespacing*{\subsection}{0in}{0.12in}{0.04in}

\titlespacing*{\section}{0in}{0.14in}{0.08in}

% don't waste space on bibliography
\setlength{\bibsep}{1.4pt}

\newenvironment{mytemize}{
\begin{itemize}
  \setlength{\itemsep}{4pt}
  \setlength{\parskip}{0pt}
  \setlength{\parsep}{0pt}
}{\end{itemize}}

\newenvironment{numlist}{
\begin{enumerate}
  \setlength{\itemsep}{4pt}
  \setlength{\parskip}{0pt}
  \setlength{\parsep}{0pt}
}{\end{enumerate}}


\newtheoremstyle{mydef}% name
	{3pt}% Space above 
	{3pt}% Space below 
	{\normalfont}% Body font
	{}% Indent amount
	{\bfseries}% Theorem head font
	{.}% Punctuation after theorem head
	{.5em}% Space after theorem head
	{\thmname{#1}\thmnumber{#2}\thmnote{#3}}% Theorem head spec (can be left empty, meaning `normal')

\newtheoremstyle{mylemthm}
	{6pt}
	{3pt}
	{\slshape}
	{}
	{\bfseries}
	{.}
	{.5em}
	{}

\renewenvironment{proof}{\noindent \emph{Proof.}}{\qed \vspace{2pt}}







\theoremstyle{mylemthm}
\newtheorem{lemma}{Lemma}
\newtheorem{theorem}{Theorem}

\theoremstyle{mydef}

%% Flush words right at end of paragraph.
%% From: http://tex.stackexchange.com/questions/16330/hfill-after-linebreak
\newcommand\rightparend[1]{{%
      \unskip\nobreak\hfil\penalty50
      \hskip2em\hbox{}\nobreak\hfil\textbf{#1}%
      \parfillskip=0pt \finalhyphendemerits=0 \par}}


\newtheorem{definition}{Def.\ }
\newtheorem{xxexample}{Ex.\ }

%% "inherent" from xxexample, but place box at the end of example.
\newenvironment{example}{
\begin{xxexample}
}{
\rightparend{$\Diamond$}
\end{xxexample}
}



