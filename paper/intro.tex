\section{Introduction}

\begin{itemize}
\item Recent years: single cores getting much more powerful,
both in terms of raw performance (insert an example), 
memory latency (insert an example), etc.
\item This has been enabled by the increasing
number of transistors per chip thanks to Moore's law. 
\item in turn has allowed for 
smart techniques which allow to exploit increasing
levels of instruction level parallelism 
\item Unfortunately, this trend can no longer
continue, for number of reasons:
\begin{itemize}
\item clock rates have reached a maximum. Wire delays
rather than tarnsistor switching are now the dominant
issue for each clock cycle. Reaching a physical limit of clock
skew as well.  
There's also the additional hig power conusmption associated
with a high clock rate (Details) 
\item Complexity of design associated with deeper pipelines,
complex branch prediciton logic and instruction level reordering
means that static and dynamic power usage is very high
\item Little instruction level parallelism to exploit.  
\end{itemize}
\item From this, there's one conclusion. Hardware developers are not 
interested simply in building better performing systems, they 
are interested in building better performing systems within a given
power budget. In that context, no further improvements for unicore
processors are likely to happen. 
\item In this context, new approach is required. The advances
in sillicon which allow multiple cores to be placed on the same
chip have encouraged new architectures to be developed
which focus on placing multiple chips per core. 
\item The rise to many-core systems, by definition either necessarily 
introduces, or allows the introduction of heterogeneity in three ways: 
\begin{itemize}
\item heterogeneity between machines: huge number of permutations
off chip on chip, ring/mesh interconnects, homogeneous cores, 
heterogeneous cores, number of cores, fully coherent or not, etc.
There's significantly more degrees of freedom on a multicore chip.
Additionally, evidence shows that parts of the system influence each
other a lot, so very tightly coupled. 
\item heterogeneity within machines: non uniform access time
to memory, IO devices, etc.
\item heterogeneity amongst processors: different ISAs, communication
\end{itemize}
\item We begin by providing a brief overview of how the various
kinds of heterogeneity manifest themselves. We then
focus on how heterogeneity is leveraged or managed for the
competing goals of power, performance, and programmability.
More specificcally, how performance and programmability can be maximised
within a given power budget.
(note: wheras innovation in architectures can unlock new lvels pf performance and or power effiency, it must be possible to achieve good performance with reasoable amunt of programming effort, if potential imrpovements are to be realised)
\end{itemize}



