\section{Introduction} 

\paragraph{}Previous years have seen a spectacular increase in 
performance of processors, thanks to a combination 
of technology advances, which allowed to place
more components on a chip, and microarchitectural 
advances (speculation, out of order execution, etc.)

\paragraph{}But this trend is now coming to a halt, for four main 
reasons: 
1) clock frequencies are reaching fundamental limit
2) we've used up pretty much all the ILP possible
3) any further increases would be too costly design wise
4) power dissipation is becoming the biggest issue. 

\paragraph{} So somehow need to find a way to continue improving 
performance, but with a fixed power budget, and without 
increasing the clock count. The solution that was
first used was to exploit a different kind of parallelism
in applications, namely thread level parallelism. This 
allowed to place more than one core on the chip (cite),
giving rise to multicore architectures. But the important thing
to note is that multicore architectures didn't arrive out of 
choice, but out of necessity to guarantee power/cost efficiency. 

\paragraph{} The core challenge with multicore architectures for software was two fold:
1) you have a lot more degrees of freedom with multicores, 
so lots of inter machine heterogeneity tat applications  now have 
to account for. 2)  they required
a significant change in paradigm, otherwise they are completely useless.

\paragraph{} As much as designing new architectures,for them to deliver the prmised performance improvements, software had to adapt. It eventually did, and now we happily use threads and paradigms like MapReduce, and, rather than try to mask multicore, happily leverage it.  

\paragraph{} But multicore still didn't deliver sufficient performance improvements whilst keeping the budget constant, so the next step was to introduce intra machine heterogeneity so that tailored componets would coexist on a chip to deliver the best performance for a specific use case. Take it to an extreme.
This lead to the rise of heteorgenous multicores. What's interesing here, is that its again the same pattern, we're adapting the computation model to gurantee power efficiency. Again, this is not currently well adapted, and existing software models appear to hide it rather than leverage it. The questionis, whether it will follow the same pattern as with multicore, or not. 

In this paper, we suvey both inter and inter machine heterogeneity... 

