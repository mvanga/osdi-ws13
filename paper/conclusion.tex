\section{Conclusion}

\paragraph{}Multicore processors exhibit significant heterogeneity, through processor or core specialisation, in a bid
to maximise the performance-power ratio. Our 
survey of heterogeneity identified a tension :
whilst computer architecture research increased
the specialisation of processors and individual cores,
often at the expense of programmibility, 
operating system and applications have generally 
hidden heterogeneity from developers. Most processors 
continue to use variants of Linux independent of
whether they are servers or desktop computers. 
Whilst modern OSes are NUMA-aware, they are not specialised for the underlying architecture.
For example, an unmodified Memcache (written for x86) was found to perform poorly when 
running on Tilera based ystems, but a tweaked Memcache (optimised for the TilePro64), 
was found to outperform the x86 version by 70\%~\cite{berezecki2011manycore}.
Inter-machine heterogeneity is thus poorly taken into account. Similarly, 
few applications are able to exploit intra-machine
heterogeneity due to a lack of consistent primitives exposing it. (TODO rephrase)


\paragraph{} Our survey thus suggests that OS and software research should focus on two directions:
firstly, inter-machine heterogeneity should be better handled by operating systems
and applications optimised for specific architectures. Secondly, operating 
systems should expose sound primitives to allow applications to reason about 
intra-machine heterogeneity. In that sense, Helios~\cite{nightingale2009helios}
 and Barrelfish's System Knowledge Base ~\cite{schupbach08embracingdiversity} are 
steps in that direction. 


