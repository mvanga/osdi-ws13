\section{Proposed Reading}
\label{sec:reading}

The key problem is in understanding the tradeoffs that such architectures
provide as well as an understanding the new bottlenecks that have been introduced. It is
increasingly clear that software for multi- and many-core architectures needs to be tailored
specifically for the underlying hardware and needs to take architectural characteristics
into account. We wish to explore research that actively considers and exploits these
characteristics at the OS level along the directions described above.

\paragraph{Understanding the hardware.}
We first intend to study the design decisions of various multi-/many-core
architectures ~\cite{Re:12,KTJR:05,BRUL:05,Va:11}, focusing on the Cell Broadband Engine, and Tilera series
processors, for massively parallel homogeneous and heterogeneous cores,
contrasting it with architectural choices made in Sandy Bridge processors.
This should help us better understand the core bottlenecks in multi-/many-
core architectures.

\paragraph{Prior work on OS/software abstractions.}
This contains two aspects: focusing on the overall OS design for multi-/many-core
systems, and looking 
at individual components, such as scheduling and memory management more
specifically. In investigating the first category, we hope to
base our initial research on the work on multikernels, fOS~\cite{WC:10}, Corey~\cite{BW08},
 Barrelfish ~\cite{SPBRBHI:08,BBDHIPRSS:09},
and on investigating alternative OS abstractions potentially better suited for many-core, 
contrasting these with  what current OSes do today~\cite{BW:10}.
  We more specifically hope to investigate
the scheduling and resource management challenges associated with large, potentially heterogeneous cores.
should scheduling be distributed? Should resources be statically partitioned? How should underlying interconnect mechanisms influence those decisions? 
We are especially interested in researching further the space partitioning
proposed by many, including Tessellation~\cite{LKBHAK:09}, as combined with aggressive thread-level speculation to leverange additional parallelism..

