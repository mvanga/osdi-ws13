\clearpage
\section{Notes}

We should probably define what
we mean by heterogeinity

\begin{itemize}
\item In~\cite{SPBRBHI:08}, define three types of heterogeneity:
\begin{itemize}
\item non-uniformity.  This refers both to memory (NUMA), but also
to communication cost within a processor
\item core diversity. Cores will have different performance, possibly
even different ISAs.
\item system diversity. Different machines will be very diverse.
\end{itemize}
\item We are specifically interested in the first two, so heterogeineity
within a single machine.
\item Non-uniformity:
\item Is heterogeneity a necessary evil that has arisen as
a result of the inability to get more performance out of single-core
and the increasigly stringent power requirements. Ak,
are current systems suffering from heterogeinity. (for ex,
in distributed systems, the use of commodity hardware is a necessary evil)
\item Or is it something which should be leveraged, and we are currently
not doing so sufficiently?
\item What are the drawbacks of current approaches?
\end{itemize}
\section{brief plan of action}
\begin{itemize}
 \item understand what the tradeoffs are
when building a heterogeneous systems
 \item come up with very cool idea
\end{itemize}
