\section{Challenges for Software}
\label{sec:challenges}

Software for multi- and many-core architectures faces many challeneges.

\subsection{Efficient Allocation of Resources}

- Scheduling seems simpler given a large enough number of cores and less than
full utilization of system resources.
- Can cores be allocated to applications? Space-wise scaling of the OS. What
about optimizing for power consumption or fairness?

\subsection{Heterogeneity}

- ISA level, performance level and memory level
- How does software deal with this heterogeneity?
- Do we gain anything from this or is it something we must "cope with"?

\subsection{Abstractions}

- Are current OS abstractions still valid for these architectures?
- Do we need to rethink anything? For example, considering a hierarchical memory
space based on access latencies? Extend or get rid of the concept of processes?

\paragraph{Summary}
The key problem seems to be in understanding the tradeoffs that such architectures
provide. Also need to consider the new bottlenecks in these designs.
