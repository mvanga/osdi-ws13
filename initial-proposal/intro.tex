\section{From unicore to multicore}

In recent decades, we saw the development of more powerful
unicore processors, with higher clock rates and more
aggressive instruction level reordering, in part
thanks to the increasing number of transistors per chip~\cite{SR97}.
 This yielded
significant performance gains. This trend is coming to and end,
 as exemplified by the scrapping of Intel's
4GHz Prescott processor. We identify
four main reasons. The increasing complexity
of design associated with deeper pipelines,
complex branch prediction logic and instruction
reordering has caused both static and dynamic power consumption
to become intolerably high. Modern uniprocessors have hit
a power wall where any further improvement would come
with too high a power cost~\cite{EH11}. Clock rates on
a single core are close to reaching fundamental hardware limits,
with wire delays and clock skew dominating within a single clock cycle ~\cite{EH11}.
Further improvements in clock frequency are thus unlikely. Thirdly, the most
complex uniprocessors exploit speculation and instruction reordering so
extensively that there is little instruction level parallelism left to
exploit ~\cite{EH11}. These hurdles, combined with advances in sillicon which allows
for multiple cores to be placed on a single die suggest a change of approach.
 New processors now contain more cores of equal (or reduced) complexity.
Multiple simpler cores instead of a highly complex one help reduce or limit power consumption,
provides designs which are simpler to verify, and allows processors
to explore other sources of parallelism, namely thread level parallelism
and data level parallelism.

The move to many/multicore was made out of necessity. In this abstract, we begin
by giving a brief overview of modern multi/manycore architectures and highlight
the challenges associated with such a significant paradigm shifts. We
seek to identify the core challenges and opportunities that we hope to focus on. 


