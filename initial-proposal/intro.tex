\section{From unicore to multicore}

In recent decades, we saw the development of more powerful
unicore processors, with higher clock rates and more
aggressive instruction level reordering, in part
thanks to the increasing number of transistors per chip~\cite{SR97}.
This yielded significant performance gains. This trend is coming to and end,
as exemplified by the cancellation of Intel's
4GHz Prescott processor due to unsustainable
power consumption~\cite{Pentium}. We identify
four main reasons. The increasing complexity
of design associated with deeper pipelines,
complex branch prediction logic and instruction-level
reordering has caused both static and dynamic power consumption
to become intolerably high. Modern uniprocessors have hit
a power wall where any further improvement would come
with too high a power cost~\cite{EH11}. Clock rates on
a single core are close to reaching fundamental hardware limits,
with wire delays and clock skew dominating within a single clock cycle ~\cite{EH11}.
Further improvements in clock frequency are thus unlikely. Thirdly, the most
complex uniprocessors exploit speculation and instruction reordering so
extensively that there is little instruction level parallelism left to
exploit ~\cite{EH11}. These hurdles, combined with advances in sillicon which allows
for multiple cores to be placed on a single die suggest a change of approach.
 New processors now contain more cores of equal (or reduced) complexity.
Multiple simpler cores instead of a highly complex one help minimise power consumption,
provides designs which are simpler to verify, and allows processors
to explore other sources of parallelism, namely thread level parallelism
and data level parallelism.

The migration to many/multicore was forced by the unstainable power consumptionof increasingly complex cores. In this paper, we perform a brief
survey of modern multi/manycore architectures in order to highlight
the challenges associated with such a significant paradigm shift and derive
a set of core questions which we intend to further investigate.

