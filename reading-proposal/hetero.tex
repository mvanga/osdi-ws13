\section{A Taxonomy of Heterogeneity}

The following texts are overview papers/textbookss which
broadly describe the area, or motivate the transition
to multicore
\begin{itemize}
\item \emph{Heterogeneous Chip Multiprocessors} General overview
of the design space and motivation for moving to multiprocessors
\item \emph{Amdahl's Law in the Multicore Era} Using Ahmdahl's law,
calculates potential speedup with regards to number of cores
in symmetric, asymmetric, and dynamic chips
\item \emph{Single-Chip Heterogeneous Computing: Does the Future Include Custom 
Logic, FPGAs, and GPGPUs?} Debates on the future 
of heterogeneous chips, specifically, just quite how diverse
are future designs going to be?
\item \emph{Scaling the bandwidth wall: challenges in and avenues for CMP scaling}
Identifies off chip memory bandwidth as the main challenge for future chip multi-
processor designs. Link to heterogeneity is a little bit more tenuous, but
provides useful insight on what future designs are likely to look like. 

\end{itemize}

\subsection{Heterogeneity in Communication}

\subsubsection{Memory Access Latency}
\begin{itemize}
\item \emph{Asymmetry-Aware Execution Placement on Manycore Chips} Focuses
specifically here on using past physical memory usage to generate better thread
placement. Focuses here specifically on true memory patterns exhibited in 
NUMA architectures
\item \emph{Traffic Management: A Holistic Approach to Memory Placement on NUMA Systems}
Makes the point that not sufficient to consider memory access latency only in scheduling
but should also take into account memory bandwdith and interconnect bottlenecks
\end{itemize}
\subsubsection{Peripheral Access Latency}
\subsubsection{Inter-processor Communication Latency}

\subsection{Heterogeneity in Individual Processors}

\subsection{Instruction Set Architecture}
\subsection{Power Consumption}
\begin{itemize}
\item \emph{Single-ISA heterogeneous multi-core architectures: the potential 
for processor power reduction}  Multiple cores on a chip with different power
footprints. Dynamically switches between all of them to match specific target.
Makes tradeoff between power and performance explicit
\item \emph {Maximizing Power Efficiency with Asymmetric Multicore Systems
} Focuses on the challenges of specialisation in asymmetric multicore systems,
more specificlaly on how to cater to thread level parallelism and instruction level
parallelism (ex: not all programs, parts of programs will exhibit the same
amount of ILP and thus benefit more or less from running on a more complex
processor)
\end{itemize}
\subsection{Processor Speed}
\begin{itemize}
\item \emph{Heterogeneous Multicores: When Slower is Faster}
Presents new operating system which can dynamically tune core
frequency, and actually show that tuning frequency down
can generate better performance on some workloads 
\end{itemize}
\subsection{Capabilities}
\subsection{Reliability and Failure Rate}
\begin{itemize}
\item \emph{It��s Time for New Programming Models for Unreliable Hardware }
Not specifically linked to heterogeneity but makes the point that, due
to smaller transistors, reliability is going to become an issue. Suggests
research directions on how to handle it. 
\end{itemize}

\section{Software}

\subsection{Applications}
\begin{itemize}
\item \emph{Merge: a programming model for heterogeneous multi-core systems}
Map-Reduce paradigm specifically geared towards heterogeneous multicore systems,
includes multiple versions of functions which are optimised for different architectures
\item \emph{Optimizing MapReduce for Multicore Architectures} Focuses on redesigning
core data structures to optimise MapReduce for multicores. Focuses on homogeneous
architectures. Objective would be to compare/contrast design decisions with
above paper when can assume homogeneity.
\item \emph{The impact of performance asymmetry in emerging multicore architectures}
Paper measures impact of performance stability on certain class of applciations.
Demonstrates that not sufficient for OS scheduler to be heterogeneity aware,
applications also need to be. 
\item \emph{Hera-JVM: a runtime system for heterogeneous multi-core 
architectures} JVM implementation for the Cell processor. Requires annotation
of parts of the code to run on the special purpose processors
\item \emph{A JVM for the Barrelfish Operating System} JVM for the Barrelfish
operating system. Claims benefits of having a single image and homogeneous view. 
Presents challenges on building this on top of non cache coherent architectures.
Focuses on message passing vs shared memory. 
\end{itemize}

\subsection{OS}
\begin{itemize}
\item \emph{Your computer is already a distributed system. Why isn't your OS?}

\item \emph{ HASS: a scheduler for heterogeneous multicore systems} Uses
core data collected offline and thread characteristics collected at runtime to 
generate near optimal per thread to core mapping
\item \emph{StarPU: a unified platform for task scheduling on heterogeneous
multicore architectures} Provides a unified runtime environment, but with the
ability for programmers to tune design decisions to exploit heterogeneinty. Demonstrates
that if tuned wel, can actually achieve superlinear performance when exploiting heterogeneity
(though is specific to HPC)
\item \emph{Asymmetry-Aware Execution Placement on Manycore Chips} Focuses
specifically here on using past physical memory usage to generate better thread
placement. Focuses here specifically on true memory patterns exhibited in 
NUMA architectures
\item \emph{Helios}
\item \emph{Single-ISA Heterogeneous Multi-Core Architectures for Multithreaded Workload Performance}
Demonstrates that through proper assignments of threads to core, heterogeneity
can actually significantly improve performance (not just power). This is not
limited to HPC. Two main reasons they highlight: more efficient die use for thread
level parallelism (can therefore exploit more TLP for the same die area), 2)
better adaptation to application diversity (different applications or parts
of applications place
different requirements on cores, heterogeneous architectures can better
taylor those requirements
\item \emph{Core fusion: accommodating software diversity in chip multiprocessors}
Groups of processors can morph/fuse into one based on application requirements. 
TODO: more detailed description. 
\item \emph{Chameleon: Operating System Support for Dynamic Processors}
Adds support to Linux to efficiently take into account dynamic processors (processors
which can reconfigure their characteristics at runtime, like Core Fusion).
Bridges gap between OS view of the world and hardware capabilities for dyn.
processors
\end{itemize}
