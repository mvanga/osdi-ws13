\section{A Taxonomy of Heterogeneity}

The following texts are overview papers/textbooks which
broadly describe the area, or motivate the transition
to multicore
\begin{itemize}
\item \emph{Programming Many Core Chips}: Textbook overview
on multicore systems with heavy focus on heterogeneity \cite{Vajda:141419}.

\item \emph{Heterogeneous Chip Multiprocessors} General overview
of the design space and motivation for moving to multiprocessors \cite{Kumar:2005:HCM:1100859.1100890}.
\item \emph{Amdahl's Law in the Multicore Era} Using Ahmdahl's law,
calculates potential speedup with regards to number of cores
in symmetric, asymmetric, and dynamic chips \cite{4563876}.
\item \emph{Single-Chip Heterogeneous Computing: Does the Future Include Custom 
Logic, FPGAs, and GPGPUs?} Debates on the future 
of heterogeneous chips, specifically, just quite how diverse
are future designs going to be? \cite{5695539}
\item \emph{Scaling the bandwidth wall: challenges in and avenues for CMP scaling}
Identifies off chip memory bandwidth as the main challenge for future chip multi-
processor designs. Link to heterogeneity is a little bit more tenuous, but
provides useful insight on what future designs are likely to look like \cite{Rogers_scalingthe}.
\item \emph{Rodinia: A Benchmark Suite for Heterogeneous Computing} Aims
to be the SPEC benchmark equivalent targeted specifically towards heterogeneous
computing (multiple CPUS, and GPUs). Provides indications of what the likely
bottlenecks ins such machines are, or likely to be with regards to application
characteristics \cite{5306797}.
\item \emph{Thousand Core Chips—A Technology Perspective} Describes the
architectural challenges that arise in a 1000 core chips, focusing
on memory bandwidth, power management and resilience \cite{Borkar:2007:TCC:1278480.1278667}.

\end{itemize}

This lists a few architectures which we believe we should study
\begin{itemize}
\item \emph{Larrabee: A Many-Core x86 Architecture for Visual Computing} \cite{4796165}
\item \emph{The Cell Processor Architecture} \cite{1540943}
\item \emph{ARM Big.LITTLE Processor Architecture} \cite{ABL}
\item \emph{Tilera} \cite{Wentzlaff:2007:OIA:1320302.1320834}
\end{itemize}

\subsection{Heterogeneity in Communication}

\begin{itemize}
\item \item \emph{Interconnections in MultiCore Architectures: understanding mechanisms,
overheads and scaling} Note. Not specifically linked to latency, but 
shows the dependencies on the interconnect to most of the rest of the system \cite{1431574}.
\end{itemize}

\subsubsection{Memory Access Latency}
\begin{itemize}
\item \emph{Asymmetry-Aware Execution Placement on Manycore Chips} Focuses
specifically here on using past physical memory usage to generate better thread
placement. Focuses here specifically on true memory patterns exhibited in 
NUMA architectures \cite{atumanov-sfma13}.
\item \emph{Traffic Management: A Holistic Approach to Memory Placement on NUMA Systems}
Makes the point that not sufficient to consider memory access latency only in scheduling
but should also take into account memory bandwidth and interconnect bottlenecks \cite{Dashti:2013:TMH:2451116.2451157}.
\item \emph{Adaptive set pinning: managing shared caches in chip multiprocessors}
Paper identifies third kind of cache misses ``intraprocessor misses'' and
investigates the implications thereof. Uses a technique called
set pinning to minimise those \cite{Srikantaiah08adaptiveset}.
\item \emph{Everything you ever wanted to know about synchronisation but
were afraid to ask} Quantifies the cost of off chip memory accesses \cite{David:2013:EYA:2517349.2522714}.
\item \emph{What Every Programmer Should Know About Memory} Textbook overview
about memory in computers, with specific chapters on non uniform memory \cite{Drepper07whatevery}.
\end{itemize}
\subsubsection{Peripheral Access Latency}

\subsubsection{Inter-processor Communication Latency}
\begin{itemize}
\item \emph{Resource Management in a Multicore Operating System} PhD dissertation
on Barrelfish. Specifically interested in the IPC procedure for this section \cite{10.1109/MM.2011.1}.

\end{itemize}
\subsection{Heterogeneity in Individual Processors}
\begin{itemize}
\item \emph{Embracing Diversity int he Barrelfish manycore operating system}
Identify three distinct types of diversity: non uniformity, core diversity and 
system diversity. Develop a knowledge base to allow people to exploit
diversity \cite{Schüpbach08embracingdiversity}.
\end{itemize}
\subsection{Instruction Set Architecture}
\subsection{Power Consumption}
\begin{itemize}
\item \emph{Single-ISA heterogeneous multi-core architectures: the potential 
for processor power reduction}  Multiple cores on a chip with different power
footprints. Dynamically switches between all of them to match specific target.
Makes tradeoff between power and performance explicit
\item \emph {Maximizing Power Efficiency with Asymmetric Multicore Systems
} Focuses on the challenges of specialisation in asymmetric multicore systems,
more specificlaly on how to cater to thread level parallelism and instruction level
parallelism (ex: not all programs, parts of programs will exhibit the same
amount of ILP and thus benefit more or less from running on a more complex
processor)
\end{itemize}
\subsection{Processor Speed}
\begin{itemize}
\item \emph{Heterogeneous Multicores: When Slower is Faster}
Presents new operating system which can dynamically tune core
frequency, and actually show that tuning frequency down
can generate better performance on some workloads 
\end{itemize}
\subsection{Capabilities}
\subsection{Reliability and Failure Rate}
\begin{itemize}
\item \emph{It¡¯s Time for New Programming Models for Unreliable Hardware }
Not specifically linked to heterogeneity but makes the point that, due
to smaller transistors, reliability is going to become an issue. Suggests
research directions on how to handle it.
\item \emph{Mixed-Mode Multicore Reliability} Identifies two classes
of applications: those that need to run on reliable hardware, and those
 that don't. Design new paradigm: Mixed Mode Multicore, which can
run normal applications without the throughput overhead of reliability
and those that requier dual modular redundancy. Presents mechanisms
for switching between the two. Note - this is not quite
reliability between processors, but rather exploiting multicore
to achieve different reliability rates.
\item \emph{Adapting to intermittent faults in multicore systems}
Focuses on how can systems adapt to varying availability of resources
caused by intermittent faults.
\end{itemize}

\section{Software}

\subsection{Applications}
\begin{itemize}
\item \emph{Merge: a programming model for heterogeneous multi-core systems}
Map-Reduce paradigm specifically geared towards heterogeneous multicore systems,
includes multiple versions of functions which are optimised for different architectures
\item \emph{Optimizing MapReduce for Multicore Architectures} Focuses on redesigning
core data structures to optimise MapReduce for multicores. Focuses on homogeneous
architectures. Objective would be to compare/contrast design decisions with
above paper when can assume homogeneity.
\item \emph{The impact of performance asymmetry in emerging multicore architectures}
Paper measures impact of performance stability on certain class of applications.
Demonstrates that not sufficient for OS scheduler to be heterogeneity aware,
applications also need to be. 
\item \emph{Hera-JVM: a runtime system for heterogeneous multi-core 
architectures} JVM implementation for the Cell processor. Requires annotation
of parts of the code to run on the special purpose processors
\item \emph{A JVM for the Barrelfish Operating System} JVM for the Barrelfish
operating system. Claims benefits of having a single image and homogeneous view. 
Presents challenges on building this on top of non cache coherent architectures.
Focuses on message passing vs shared memory. 
\item \emph{Understanding Diversity int he Barrelfish manycore operating system}
Identify three distinct types of diversity: non uniformity, core diversity and 
system diversity. Develop a knowledge base to allow people to exploit
diversity
\end{itemize}

\subsection{OS}
\begin{itemize}
\item \emph{Your computer is already a distributed system. Why isn't your OS?}
\item \emph{Resource Management in the Tessellation Manycore OS}
\item \emph{The Multikernel: A new OS architecture for scalable multicore systems}
\item \emph{An Operating System for Multicore and Clouds: Mechanisms and Implementation}
\item \emph{Corey: An Operating System for Many Cores}
\item \emph{Helios: Heterogeneous Multiprocessing with Satellite Kernels}
\item \emph{Tornado: Maximizing Locality and Concurrency in a Shared
Memory Multiprocessor Operating System}


\item \emph{ HASS: a scheduler for heterogeneous multicore systems} Uses
core data collected offline and thread characteristics collected at runtime to 
generate near optimal per thread to core mapping
\item \emph{Accelerating Critical Section Execution with Asymmetric Multi-Core Architectures}
Critical sections are serialization points. Transfer thread to high performance 
core during those.

\item \emph{StarPU: a unified platform for task scheduling on heterogeneous
multicore architectures} Provides a unified runtime environment, but with the
ability for programmers to tune design decisions to exploit heterogeneity. Demonstrates
that if tuned well, can actually achieve superlinear performance when exploiting heterogeneity
(though is specific to HPC)
\item \emph{Asymmetry-Aware Execution Placement on Manycore Chips} Focuses
specifically here on using past physical memory usage to generate better thread
placement. Focuses here specifically on true memory patterns exhibited in 
NUMA architectures
\item \emph{Single-ISA Heterogeneous Multi-Core Architectures for Multithreaded Workload Performance}
Demonstrates that through proper assignments of threads to core, heterogeneity
can actually significantly improve performance (not just power). This is not
limited to HPC. Two main reasons they highlight: more efficient die use for thread
level parallelism (can therefore exploit more TLP for the same die area), 2)
better adaptation to application diversity (different applications or parts
of applications place
different requirements on cores, heterogeneous architectures can better
tailor those requirements
\item \emph{Challenges and Opportunities
in Many-Core Computing} As long as provide the required abstractions for applications, 
many core computing and heterogeneity provides significant opportunities for application
speedups and better tailoring of what the applications want and what the OS can provide
\item \emph{Core fusion: accommodating software diversity in chip multiprocessors}
Groups of processors can morph/fuse into one based on application requirements. 
TODO: more detailed description. 
\item \emph{Chameleon: Operating System Support for Dynamic Processors}
Adds support to Linux to efficiently take into account dynamic processors (processors
which can reconfigure their characteristics at runtime, like Core Fusion).
Bridges gap between OS view of the world and hardware capabilities for dyn.
processors
\item \emph{Using OS Observations to improve performance in multicore systems}
Claim that necessary for OS to leverage past information of thread performance
to adapt future policies. Identify main problems to be cache interference in the LLC,
lack of intelligent thread migration, and lack of adaption to different
core characteristics. 
\item \emph{An Asymmetric Distributed Shared Memory Model for Heterogeneous Parallel Systems}
Seems specifically targeted towards CPU vs GPUS. But makes the claim for an asymmetric
memory space, where certain devices can access memory objects within he same address space,
but not others. 
\item \emph{An Analysis of Linux Scalability to Many Cores} Most scalability bottlenecks
occur because of how applications are designed, or how they use kernel datastructures.
Can achieve very good scalability if modify the kernel only slightly 
\end{itemize}
