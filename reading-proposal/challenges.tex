\section{Challenges for Software}
\label{sec:challenges}

The vast variety of multi- and many-core architectures available pose many challenges for
software (both at the operating system and application level). We wish to explore these
challenges along three directions.

\paragraph{Efficient Allocation of Resources}

Resource allocation is one of the primary roles of the operating system. The infusion of a
large number of cores makes the job of the OS harder. In particular, the decision making
is complicated as constraints such as minimizing power consumption and fairness of allocation
are introduced. That is, the resource utilization becomes non-linear in the number of cores due to
inefficient decision making at the software level.

\paragraph{Non-Uniformity and Heterogeneity}

Due to the increasing physical distance between cores in multi-core architectures, memory
access latency is non-uniform across cores and interconnect bandwidth limits
introduce additional non-uniformity due to interference from other cores.

In addition, ISA-level and performance variations across cores introduce a level of
heterogeneity into the system. It is unclear what the tradeoffs are with this added
heterogeneity it is unclear how to best exploit this at the operating system level. How
should this heterogeneity be abstracted? How much of it should be exposed to user
applications?

\paragraph{OS Abstractions}

With these complexities in mind, the question becomes whether state-of-the-art
OS structure and abstractions are still optimal, or whether they can be better designed.
A number of research OSes investigate this question, including Barrelfish, Helios, Tesselation, Corey, fOs.
Finally, what are the common trends among these approaches, and where can they be improved?

%- Are current OS abstractions still valid for these architectures?
%- Do we need to rethink anything? For example, considering a hierarchical memory
%space based on access latencies? Extend or get rid of the concept of processes?

%- key issue that arises in multicore architecture
%is the sharing of resources which wasn't there previously
%That can come up in the form of sharing global datastructures
%(page tables, etc.), interference (TLB shootdown, cache pollution)
%to physical resources (interconnect bottlenecks).
%- Make the claim that whereas previously memory latency was the
%problem, memory bandwidth is now a bigger issue (The ASPLOS
%paper makes that point well)
%- Amdhal's law states that any parallel speedup is limited by
%the serial performance of the application, how do we guarantee
%good serial performance if we only have very simple cores.
%- Applications. (this is linked to abstractions, maybe?) How do
%we design applications which are highly parallel, sufficiently
%so that they can be run on a large number of cores efficiently,
%or alternatively, what techniques can we use to
%implicitly parallelise such applications to be able to make them
%run on multiple cores (speculation comes in to play here)
